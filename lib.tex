\documentclass[
  bibliography=totoc,     % Literatur im Inhaltsverzeichnis
  captions=tableheading,  % Tabellenüberschriften
  titlepage=firstiscover, % Titelseite ist Deckblatt
  parskip=half,
]{scrartcl}
\usepackage[utf8]{inputenc}

% Warnung, falls nochmal kompiliert werden muss
\usepackage[aux]{rerunfilecheck}

% Erweiterungen für amsmath
\usepackage{mathtools}

% Fonteinstellungen
\usepackage{fontspec}
% Latin Modern Fonts werden automatisch geladen
% Alternativ zum Beispiel:
%\setromanfont{Libertinus Serif}
%\setsansfont{Libertinus Sans}
%\setmonofont{Libertinus Mono}

% Wenn man andere Schriftarten gesetzt hat,
% sollte man das Seiten-Layout neu berechnen lassen
\recalctypearea{}

% deutsche Spracheinstellungen
\usepackage[ngerman]{babel}


\usepackage[
  math-style=ISO,    % ┐
  bold-style=ISO,    % │
  sans-style=italic, % │ ISO-Standard folgen
  nabla=upright,     % │
  partial=upright,   % ┘
  warnings-off={           % ┐
    mathtools-colon,       % │ unnötige Warnungen ausschalten
    mathtools-overbracket, % │
  },                       % ┘
]{unicode-math}

% traditionelle Fonts für Mathematik
\setmathfont{Latin Modern Math}
% Alternativ zum Beispiel:
%\setmathfont{Libertinus Math}

\setmathfont{XITS Math}[range={scr, bfscr}]
\setmathfont{XITS Math}[range={cal, bfcal}, StylisticSet=1]

% Zahlen und Einheiten
\usepackage[
  locale=DE,                   % deutsche Einstellungen
  separate-uncertainty=true,   % immer Unsicherheit mit \pm
  per-mode=symbol-or-fraction, % / in inline math, fraction in display math
]{siunitx}
\sisetup{output-decimal-marker = {.}}

% chemische Formeln
\usepackage[
  version=4,
  math-greek=default, % ┐ mit unicode-math zusammenarbeiten
  text-greek=default, % ┘
]{mhchem}

% richtige Anführungszeichen
\usepackage[autostyle]{csquotes}

% schöne Brüche im Text
\usepackage{xfrac}

% Standardplatzierung für Floats einstellen
\usepackage{float}
\floatplacement{figure}{htbp}
\floatplacement{table}{htbp}

% Floats innerhalb einer Section halten
\usepackage[
  section, % Floats innerhalb der Section halten
  below,   % unterhalb der Section aber auf der selben Seite ist ok
]{placeins}

% Seite drehen für breite Tabellen: landscape Umgebung
\usepackage{pdflscape}

% Captions schöner machen.
\usepackage[
  labelfont=bf,        % Tabelle x: Abbildung y: ist jetzt fett
  font=small,          % Schrift etwas kleiner als Dokument
  width=0.9\textwidth, % maximale Breite einer Caption schmaler
]{caption}
% subfigure, subtable, subref
\usepackage{subcaption}

% schöne Tabellen
\usepackage{booktabs}

% Verbesserungen am Schriftbild
\usepackage{microtype}

% Hyperlinks im Dokument
\usepackage[
  german,
  unicode,        % Unicode in PDF-Attributen erlauben
  pdfusetitle,    % Titel, Autoren und Datum als PDF-Attribute
  pdfcreator={},  % ┐ PDF-Attribute säubern
  pdfproducer={}, % ┘
]{hyperref}
% erweiterte Bookmarks im PDF
\usepackage{bookmark}

% Trennung von Wörtern mit Strichen
\usepackage[shortcuts]{extdash}

% Paket float verbessern
\usepackage{scrhack}

\publishers{TU Dortmund – Fakultät Physik}

\usepackage{esint} % command: \oiint
\usepackage{amsmath} % environment: align, align*, more.. !
%\usepackage{amssymb} % command: \mathbb % XXX: already loaded by lib-tudo.tex?
\usepackage{physics} % command: \vectorbold, \dd, \dv, ... lots!
\AtBeginDocument{\RenewCommandCopy\qty\SI} % Clara möchte qty verwenden!
\usepackage{graphicx} % command: \includegraphics
\usepackage{datetime} % command: currenttime
\usepackage{svg} % command: \includesvg
\usepackage{xstring} % command: \IfEqCase
\usepackage{xcolor} % command: \color
\usepackage{tabularx} % environment: tabularx
\usepackage{longtable} % environment: longtable <- it has page breaks..
\usepackage{wrapfig} % environment: wrapfigure
%\usepackage{subfigure} % environment: wrapfigure
%\usepackage[nobottomtitles*]{titlesec} % avoid page break directly after headlines % FIXME: TU-Do template conflicts with this!
%\renewcommand{\bottomtitlespace}{0.25\textheight}
\usepackage[backend=biber,sorting=none]{biblatex} % sorting=none -> sortiert nach reihenfolge der zitierung im text
%\usepackage{geometry} % command: \geometry
%\geometry{bottom=3cm}
% notes:
%\newcommand{\withoutindent}[1]{\setlength{\parindent}{0pt}#1\setlength{\parindent}{1.5em}}
\newcommand{\withoutindent}[1]{#1}
\newcommand{\note}[1]{{\it{}Hinweis:} #1}
\newcommand{\Q}[1]{\fbox{\parbox{\textwidth}{{\it{}Frage:} #1}}}
\newcommand{\err}[1]{\withoutindent{\\{\it{}Fehler:} #1}}
\newcommand{\TODO}[1]{\withoutindent{{\color{red}\textbf{TODO}: #1}}}
% usage: \figureWithWidth {1: path-to-img (relative to cwd)} {2: caption} {3: width, e.g. 1, 0.6, ..} {4: optional post fix, e.g. \label{fig:xy}} {5: image type, e.g. png, svg}
\newcommand{\figureWithWidth}[5]{%
\begin{figure}[ht!]
  \centering
  \IfEqCase{#5}{
    {png}{\includegraphics[width=#3\textwidth]{#1}}
    {pdf}{\includegraphics[width=#3\textwidth]{#1}}
    {svg}{\includesvg[width=#3\textwidth]{#1}}
  }[\PackageError{figureWithWidth}{Undefined image type}]
  \caption{#2}
  #4
\end{figure}%
}
% usage: \plot* {1: path} {2: caption} {3: label}
\newcommand{\plotSVG}[3]{\figureWithWidth{#1}{#2}{1.0}{\label{#3}}{svg}}
\newcommand{\plotPDF}[3]{\figureWithWidth{#1}{#2}{1.0}{\label{#3}}{pdf}}
\newcommand{\plotPNG}[3]{\figureWithWidth{#1}{#2}{1.0}{\label{#3}}{png}}
\newcommand{\plotPNGSmall}[3]{\figureWithWidth{#1}{#2}{0.6}{\label{#3}}{png}}
\newcommand{\plotPNGTiny}[3]{\figureWithWidth{#1}{#2}{0.3}{\label{#3}}{png}}
\newcommand{\plotPNGWrap}[3]{
\begin{wrapfigure}{r}{0.35\textwidth}
  \begin{center}
    \includegraphics[width=0.3\textwidth]{#1}
  \end{center}
  \caption{#2}
  \label{#3}
\end{wrapfigure}
}
% usage: \includePdf{<path>}{<caption>}{<label>}{<page>}
\newcommand{\includePdf}[4]{
%\newpage
\begin{figure}[ht!]
\centering
\includegraphics[width=0.8\textwidth,page=#4]{#1}
\caption{#2}
\label{#3}
\end{figure}
%\newpage
}
% doing math:
\newcommand{\EQ}[1]{\begin{align*}#1\end{align*}}
\newcommand{\EQNUM}[1]{\begin{align}#1\end{align}}
% inside math:
\newcommand{\EQU}{\quad\Leftrightarrow\quad}
\newcommand{\mbeq}{\overset{!}{=}}
\newcommand{\expl}[1]{\ \Big\vert\ \text{#1}\ }
\newcommand{\expli}[1]{\quad \text{#1}\ }
\newcommand{\mat}[1]{\vectorbold{#1}}
\newcommand{\vv}[2]{\begin{pmatrix}#1\\#2\end{pmatrix}}
\newcommand{\vvv}[3]{\begin{pmatrix}#1\\#2\\#3\end{pmatrix}}
\newcommand{\vvvv}[4]{\begin{pmatrix}#1\\#2\\#3\\#4\end{pmatrix}}
\renewcommand{\phi}{\varphi}
\newcommand{\Nabla}{\vvv{\pdv{x}}{\pdv{y}}{\pdv{z}}} % physics package rocks!
\newcommand{\nablav}{\vec{\nabla}}
\newcommand{\equcuz}[1]{\stackrel{#1}{=}}
\newcommand{\vnorm}[1]{\left\lVert#1\right\rVert}
\newcommand{\Set}[1]{\left\{#1\right\}}
\newcommand{\Par}[1]{\left(#1\right)}
\newcommand{\Parc}[1]{\left[#1\right]}
\newcommand{\Parb}[1]{\big(#1\big)}
\newcommand{\vmid}{\ \middle|\ }
\newcommand{\dist}[1]{\overline{#1}}
\newcommand{\Unitx}[1]{\,\si{#1}}
\newcommand{\Unit}[1]{\,\text{#1}}
\newcommand{\UnitFrac}[2]{\,\frac{\text{#1}}{\text{#2}}}
\newcommand{\degC}{\,^\circ\text{C}}
\renewcommand{\deg}{\,^\circ}
\newcommand{\Expected}[1]{\left\langle#1\right\rangle}
% other:
\newcommand{\enum}[1]{\begin{enumerate}#1\end{enumerate}}
% \tableAny {caption} {layout} {label} {content}
\newcommand{\tableAny}[4]{
\begin{longtable}{#2}\caption{#1}\label{#3}#4\hline
\end{longtable}
}
\iflanguage{english}{
    \newcommand{\headerAnyTableContinued}{Continued}
}{
\iflanguage{german}{
    \newcommand{\headerAnyTableContinued}{Fortsetsung}
}{}
}
\newcommand{\headerAny}[1]{%
\\\hline#1\\\hline
\endfirsthead
\caption[]{(\headerAnyTableContinued)}
\\\hline#1\\\hline
\endhead
}
\newcommand{\entryAny}[1]{#1\\}
% formatting:
\iflanguage{german}{
    \renewcommand{\figurename}{Abbildung}
}{}
% praktikum:
\addbibresource{literature.bib}
\newcommand{\header}{
  \maketitle
  \thispagestyle{empty}
  \tableofcontents
  \newpage
}

\iflanguage{english}{
    % english
    \newcommand{\Zielsetzung}{\section{Goal}\label{sec:zielsetzung}}
    \newcommand{\Theorie}{\section{Theory}\label{sec:theorie}}
    \newcommand{\Aufgabe}{\section{Task}\label{sec:aufgabe}}
    \newcommand{\Vorbereitung}{\section{Preparation}\label{sec:vorbereitung}}
    \newcommand{\VersuchsaufbauUndDurchfuehrung}{\section{Experimental Setup and Experimental Procedure}\label{sec:aufbaudurchfuehr}}
    \newcommand{\Versuchsaufbau}{\section{Experimental Setup}\label{sec:aufbau}}
    \newcommand{\Versuchsdurchfuehrung}{\section{Experimental Procedure}\label{sec:durchfuehrung}}
    \newcommand{\Messwerte}{\section{Measurements}\label{sec:messwerte}}
    \newcommand{\Auswertung}{\section{Analysis}\label{sec:auswertung}}
    \newcommand{\Diskussion}{\section{Discussion}\label{sec:diskussion}}
    \newcommand{\Literatur}{\newpage{}\printbibliography{}}
    \newcommand{\Anhang}{\section*{Appendix}\label{sec:anhang}}
    \newcommand{\initVersuchAm}[1]{
    \subject{\experimentNo}
    \title{V\experimentNo: \experimentName}
    \date{%
        Experiment: #1
        \hspace{1em}
        Submission: \today
    }
    }
}{
\iflanguage{german}{
    % german
    \newcommand{\Zielsetzung}{\section{Zielsetzung}\label{sec:zielsetzung}}
    \newcommand{\Theorie}{\section{Theorie}\label{sec:theorie}}
    \newcommand{\Aufgabe}{\section{Aufgabe}\label{sec:aufgabe}}
    \newcommand{\Vorbereitung}{\section{Vorbereitung}\label{sec:vorbereitung}}
    \newcommand{\VersuchsaufbauUndDurchfuehrung}{\section{Versuchsaufbau und Durchführung}\label{sec:aufbaudurchfuehr}}
    \newcommand{\Versuchsaufbau}{\section{Versuchsaufbau}\label{sec:aufbau}}
    \newcommand{\Versuchsdurchfuehrung}{\section{Durchführung}\label{sec:durchfuehrung}}
    \newcommand{\Messwerte}{\section{Messwerte}\label{sec:messwerte}}
    \newcommand{\Auswertung}{\section{Auswertung}\label{sec:auswertung}}
    \newcommand{\Diskussion}{\section{Diskussion}\label{sec:diskussion}}
    \newcommand{\Literatur}{\newpage{}\printbibliography{}}
    \newcommand{\Anhang}{\section*{Anhang}\label{sec:anhang}}
    \newcommand{\initVersuchAm}[1]{
    \subject{\experimentNo}
    \title{V\experimentNo: \experimentName}
    \date{%
        Durchführung: #1
        \hspace{1em}
        Abgabe: \today
    }
    }
}{}
}
